\documentclass[../report.tex]{subfiles}

\begin{document}

\chapter{Abstract}

This supervised learning project develops a theory \& computation of lagrangian coherent structures using 2 approaches i.e., one is \textit{lagrangian approach} and the other is \textit{eularian approach}. Both these approaches have their own advantages and disadvantages. We will go through their approaches and deep mathematical relationship that exists between two of them and we will define new Eularian diagnostic: Infinitesimal-time LCS (iLCS). iLCS will be shown to be the limit of LCS as \(t \rightarrow 0\) and finally using the iLCS we will demonstrate the effectiveness of iLCS using double gyre and comparing the alteration rate field to F.I.L.E. field.

\end{document}