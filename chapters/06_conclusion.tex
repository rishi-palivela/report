\documentclass[../report.tex]{subfiles}

\begin{document}

\chapter{Conclusion}

These mathematical connections prove that the attraction and repulsion rates are the limits of F.I.L.E. as integration time goes to 0. Additionally this manuscript also shows that for small integration time \(|T| << 1\) eigen vectors for cauchy-green strain tensor are equal to eularian rate of strain tensor.

\section{Equality of Eigen Vectors of \(S\) \& \(C\)}
\begin{equation}
  \begin{aligned}
    S\epsilon_i &= S_i\epsilon_i \\
    2iS\epsilon_i + \epsilon_i &= 2T_{S_i}\epsilon_i + \epsilon_i \\
    (2iS + 1)\epsilon_i &= (2iS_i + 1)\epsilon_i \\
    C\epsilon_i &= \lambda_i\epsilon_i
  \end{aligned}
\end{equation}

\(\Rightarrow \epsilon_i\) is an eigen vector of \(S\) then \(\epsilon_i\) is an eigen vector of \(C\).

\(\therefore\) We can say as \(T\) goes to 0, eigen vector of \(C\) is same as eigen vector of \(S\).
\end{document}