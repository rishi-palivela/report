\documentclass[../presentation.tex]{subfiles}

\begin{document}

% \section{Literature Review}
% \subsection{Lagrangian Approach}

\begin{frame}
  \frametitle{Literature Review - Lagrangian Approach}

  \begin{itemize}
    \item Consider the Dynamical System,
    \begin{equation}
      \begin{aligned}
        \frac{d}{dt}(x(t)) &= V(x(t), t) \\
        x_0 &= x(t_0) \\
        x \in \mathbb{R}^n &, t \in \mathbb{R}
      \end{aligned}
    \end{equation}
  
    \item For LCS, it is needed to compute Finite-Time-Lyapunov-Exponent (FTLE), with a FTLE for each grid point the structures can be plotted.
    
    \item FTLE is a scaler \(\sigma_{t_0}^T(n)\) which represents structures of a fluid at a location. The maxima show the attracting (or repelling barriers). Let's say a particle at \(x(t_0)\) goes to a new location after time \(T\). 
  \end{itemize}
\end{frame}

\begin{frame}
  \frametitle{Literature Review - Lagrangian Approach}

  \begin{itemize}
    \item Flow map of that point can be written as
    \begin{equation}
      F_{t_0}^t(x_0) = x_0 + \int_{t_0}^{T} V(x(t), t)\,dt
    \end{equation}

    \item Let's say there is another point close to \(x(t_0)\) which is \(y = x + \delta x(t_0)\). After a time interval \(T\), distance between these 2 points becomes
    \begin{equation}
      \begin{aligned}
        \delta x(t_0 + T) &= F_{t_0}^{t_0 + T} (y) - F_{t_0}^{t_0 + T} (x) \\
        &= \nabla F_{t_0}^{t_0 + T}(x)
      \end{aligned}
    \end{equation}

    \item From above equation we can calculate the neon strain tensor,
    \begin{equation}
      \begin{aligned}
        C &= \nabla {F_{t_0}^{t_0 + T}(x)}^T \cdot \nabla F_{t_0}^{t_0 + T}(n)
      \end{aligned}
    \end{equation}
  \end{itemize}
\end{frame}

\begin{frame}
  \frametitle{Literature Review - Lagrangian Approach}

  \begin{itemize}
    \item Eigan values of which are \(\lambda_1, \lambda_2, \dots, \lambda_n\) \& associated normalized eigenvectors, \(\epsilon_{\lambda_i}\;\; (i \in \{1, 2, 3, \dots, n\})\)

    \item \(\therefore\) From the maximum eigenvalue of Chuchy-Green tensor FTLE can be calculated as
    \begin{itemize}
      \item \begin{equation}
        \begin{aligned}
          \sigma_{t_0}^T (x_0) = \frac{1}{2|T|} \log(\lambda_n)
        \end{aligned}
      \end{equation}
      
      \item \begin{equation}
        \begin{aligned}
          \lambda_n &= max\; eig (C) \\
          &= max\; eig(\nabla {F_{t_0}^{t_0 + T}(x)}^T \cdot \nabla F_{t_0}^{t_0 + T}(n))
        \end{aligned}
      \end{equation}
    \end{itemize}
  \end{itemize}
\end{frame}

\begin{frame}
  \frametitle{Literature Review - Lagrangian Approach}

  \begin{itemize}
    \item To complete FTLE, it is necessary to have locations of particles at initial state \(t = t_0\) \& at \(t = t_0 + T\). \(\therefore\) Flow map can be determined.
  \end{itemize}

  {\tiny
  \begin{equation}
    \begin{aligned}
      {F_{t_0}^{t_0 + T}} &= \begin{bmatrix}
        \frac{x_{i + 1, j}(t_0 + T) - x_{i - 1, j}(t_0 + T)}{x_{i+1, j}(t_0) - x_{i-1, j}(t_0)} & \frac{x_{i, j+1}(t_0 + T) - x_{i,j-1}(t_0 + T)}{y_{i,j+1}(t_0) - y_{i,j-1}(t_0)} \\[12pt]
        
        \frac{y_{i+1, j}(t_0 + T) - y_{i-1, j}(t_0 + T)}{x_{i+1,j}(t_0) - x_{i-1, j}(t_0)} & \frac{y_{i,j+1}(t_0 + T) - y_{i, j - 1}(t_0 + T)}{y_{i, j+1}(t_0) - y_{i, j-1}(t_0}
      \end{bmatrix}
    \end{aligned}
  \end{equation}
  }

  \begin{figure}[H]
    \centering
    \includegraphics[width=0.8\linewidth]{images/image_1.png}
    \caption{Flowmap used for computing the FTLE}
    \label{fig:fig-1}
  \end{figure}
\end{frame}

% \subsection{Eulerian Approach}
\begin{frame}
  \frametitle{Literature Review - Eulerian Approach}

  \begin{itemize}
    \item The Eulerian role of strain Tensor is defined as
    \begin{equation}
      \begin{aligned}
        S(x, t) = \frac{1}{2}{\nabla V(x, t) + \nabla V(x, t)^T}
      \end{aligned}
    \end{equation}
    
    \item And eigan values of which are \(s_1 < s_2 < \dots < s_n\) and associated normalized eigen vectors, \(\epsilon_{S_i}\;\; i \in \{1, 2, \dots, n\}\)
    
    \item From the eigen values of eulerian rate of stain tensor, one can identify regions of flow which are more attracting \& repelling.
    
    \item There \(s_1\) \rightarrow minimum eigen values provides measure of attraction \& \(s_2\) maximum value of repulsion.
  \end{itemize}
\end{frame}

% \subsubsection{Relation between Cauchy-Green Strain Tensor \& Eulerian Rate of Strain Tensor}
\begin{frame}
  \frametitle{\small Rel. b/w Cauchy-Green Strain Tensor \& Eulerian Rate of Strain Tensor}

  \begin{itemize}
    \item Eigen value of \(S\) as FTLE limit as integration of time goes to 0. \par

    \item For small \(|T|\), let us expand  \(C_{t_0}^t (x)\) as
    \begin{equation}
      \begin{aligned}
        C_{t_0}^t (n) &= 1 + 2T S(x, t_0) + T^2 B(x, t_0) + \frac{1}{2} T^3 Q(x, t_0) + O(T^4)
      \end{aligned}
    \end{equation}
  
    \item Where,
    \begin{equation}
      \begin{aligned}
        B(x, t_0) &= \frac{1}{2}[\nabla a(x, t_0) + (\nabla a(x, t_0))^T] + \nabla V(x, t_0)^T \cdot \nabla V(x, t_0)
      \end{aligned}
    \end{equation}
    
    \item Where acceleration field \(a(x, t_0)\) is
    \begin{equation}
      \begin{aligned}
        a(x, t_0) = \frac{d}{dt}V(x, t_0) = \frac{\partial}{\partial t} V(x, t_0) + V(x, t_0) \cdot \nabla V(x, t_0)
      \end{aligned}
    \end{equation}
    \begin{equation}
      \begin{aligned}
        \lambda_n &= \lambda^+ (C_{t_0}^t (x)) \text{for small, } T > 0
      \end{aligned}
    \end{equation}
  \end{itemize}
\end{frame}

\begin{frame}
  \frametitle{\small Rel. b/w Cauchy-Green Strain Tensor \& Eulerian Rate of Strain Tensor}

  \begin{itemize}
    \item We can neglect \(O(T^2)\) in
    \begin{equation}
      \begin{aligned}
        \therefore \lambda^+ (C_{t_0}^t (x)) &= 1 + 2T \lambda^+ (S(x, t_0)) + O(T^2)
      \end{aligned}
    \end{equation}
    
    \item \begin{equation}
      \begin{aligned}
        \log(\lambda_n) &= \log(1 + 2T\lambda^+S(x, t_0)) \\
        &= 2T\lambda^+(S(x, t_0)) \\
        &= 2Ts_n(x, t)
      \end{aligned}
    \end{equation}
    
    \item In the limit of small \(T,\; \log(1+\epsilon) = \epsilon\)
    \begin{equation}
      \begin{aligned}
        \sigma_{t_0}^T &= \frac{1}{2|T|}\log(\lambda_n) \\
        &= \frac{1}{2T} \cdot 2T \cdot S(x, t_0) \\
        &= s_n(x, t_0)
      \end{aligned}
    \end{equation}
  \end{itemize}
\end{frame}

\begin{frame}
  \frametitle{\small Rel. b/w Cauchy-Green Strain Tensor \& Eulerian Rate of Strain Tensor}

  \begin{itemize}
    \item For \(T < 0\), with small T
    \begin{equation}
      \begin{aligned}
        \lambda^+ (C_{t_0}^t(x)) &= 1 + 2T \lambda^- (S(x, t_0))
      \end{aligned}
    \end{equation}
    
    \begin{equation}
      \begin{aligned}
        \log(\lambda_n) &= 2T\lambda^- (S(x, t_0)) \\
        &= 2Ts_1(x, t_0)
      \end{aligned}
    \end{equation}

    \item \(\therefore |T| = -T \text{if } T < 0\)
    \begin{equation}
      \begin{aligned}
        \sigma_{t_0}^t &= \frac{1}{2|T|} \log(\lambda_n) = -s_1(x, t_0)
      \end{aligned}
    \end{equation}
  \end{itemize}
\end{frame}

\begin{frame}
  \frametitle{\small Rel. b/w Cauchy-Green Strain Tensor \& Eulerian Rate of Strain Tensor}

  \begin{itemize}
    \item \(\therefore\) We can summarize as follows,
  \end{itemize}
  \begin{equation}
    \begin{aligned}
      \sigma_{t_0}^t &= \pm s^\pm (x, t_0)\; \text{as } t - t_0 \rightarrow 0^\pm
    \end{aligned}
  \end{equation}
  
  \begin{equation}
    \nabla V = \begin{bmatrix}
      \frac{\partial U}{\partial x} & \frac{\partial U}{\partial y} \\[12pt]
      \frac{\partial V}{\partial x} & \frac{\partial V}{\partial y} \\
    \end{bmatrix}
  \end{equation}
  
  \begin{equation}
    \nabla S = \begin{bmatrix}
      \frac{\partial U}{\partial x} & \frac{1}{2} (\frac{\partial U}{\partial y} + \frac{\partial V}{\partial x}) \\[12pt]
      \frac{1}{2} (\frac{\partial U}{\partial y} + \frac{\partial V}{\partial x}) & \frac{\partial V}{\partial y} \\
    \end{bmatrix}
  \end{equation}
\end{frame}

% \subsubsection{Equality of Eigen Vectors of \(S\) \& \(C\)}
\begin{frame}
  \frametitle{\small Rel. b/w Cauchy-Green Strain Tensor \& Eulerian Rate of Strain Tensor}

  \begin{itemize}
    \item \begin{equation}
      \begin{aligned}
        S\epsilon_i &= S_i\epsilon_i \\
        2TS\epsilon_i + \epsilon_i &= 2TS_i\epsilon_i + \epsilon_i \\
        (2TS + 1)\epsilon_i &= (2TS_i + 1)\epsilon_i \\
        C\epsilon_i &= \lambda_i\epsilon_i
      \end{aligned}
    \end{equation}
    
    \item \(\epsilon_i\) is an eigen vector of \(S\) then \(\epsilon_i\) is an eigen vector of \(C\).
    
    \item \(\therefore\) We can say as \(T\) goes to 0, eigen vector of \(C\) is same as eigen vector of \(S\).
  \end{itemize}
\end{frame}
\end{document}